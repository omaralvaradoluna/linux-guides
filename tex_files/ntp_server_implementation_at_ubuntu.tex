\documentclass{article}
\usepackage[spanish]{babel}
\selectlanguage{spanish}
\usepackage[utf8]{inputenc}
\usepackage{fancyhdr} 
\pagestyle{fancy} 
\author{Omnestek (c) 2016}
\title{Implementaci\'on de servidor ntp en Ubuntu}
\lhead{Omnestek (c) 2016}
\begin{document}
\maketitle
\section{Configuraci\'on del Servidor}
Lo primero que se debe realizar es la configuraci\'on del servidor de ntp (Network Time Protocol) para lo cual utilizaremos apt-get install ntp
\begin{verbatim}
$ sudo apt-get install ntp
\end{verbatim}
Despu\'es de \'esto debemos editar el archivo /etc/ntp.conf para ajustar los servidores a los que se les har\'an las peticiones en primer lugar.

Como siguiente paso debemos verificar la configuraci\'on de nuestra red

\begin{verbatim}
$ ifconfig
	  eth0      Link encap:Ethernet  direcci\'onHW b8:ae:ed:94:60:10  
          Direc. inet:192.168.1.XXX  Difus.:192.168.1.255  M\'asc:255.255.255.0
          Direcci\'on inet6: fe80::baae:edff:fe94:6010/64 Alcance:Enlace
          ACTIVO DIFUSI\'ONN FUNCIONANDO MULTICAST  MTU:1500  M\'etrica:1
          Paquetes RX:155197238 errores:0 perdidos:22093 overruns:0 frame:0
          Paquetes TX:145506445 errores:0 perdidos:0 overruns:0 carrier:0
          colisiones:0 long.colaTX:1000 
          Bytes RX:22398083673 (22.3 GB)  TX bytes:30080215169 (30.0 GB)
\end{verbatim}
Como podemos ver, tenemos una red configurada a 3 octetos. Entonces agregaremos las siguientes l\'inea al archivo de configuraci\'on ntp.conf
\begin{verbatim}
restrict 192.168.1.0 mask 255.255.255.0 nomodify notrap
# If you want to provide time to your local subnet, change the next line.
# (Again, the address is an example only.)
broadcast 192.168.1.255
\end{verbatim}
Salir y reiniciar el servicio
\begin{verbatim}
$ /etc/init.d/ntp restart
\end{verbatim}
Con \'esto ya tenemos listo el servidor para que sea consultado por otros ordenadores y podemos ver su estatus con el siguiente comando
\begin{verbatim}
Time100.Stupi.S .PPS.            1 u    4   64    1  227.401   14.989   0.000
 juniperberry.ca 193.79.237.14    2 u    3   64    1  183.433    7.461   0.000
 192.168.1.255   .BCST.          16 u    -   64    0    0.000    0.000   0.000
 \end{verbatim}
\section{Configuraci\'on de los Clientes}
\subsection{Linux}
Para un ordenador con Linux hay que instalar tambi\'en el paquete ntp con apt-get y luego editar el archivo /etc/ntp.conf
\begin{verbatim}
server 192.168.1.XXX prefer iburst
\end{verbatim}
Salir y reiniciar el servicio
\begin{verbatim}
$ /etc/init.d/ntp restart
\end{verbatim}
\subsection{Windows}
Para configurar en una m\'aquina con Windows hay que dar doble click en la hora y luego en la pesta\~na de Hora de Internet ingresar la direcci\'on del servidor een el campo servidor.
\section{Referencias}
\begin{enumerate}
\item http://askubuntu.com/questions/14558/how-do-i-setup-a-local-ntp-server
\item http://blogging.dragon.org.uk/setting-up-ntp-on-ubuntu-14-04/
\end{enumerate}
\end{document}